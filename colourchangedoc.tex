\documentclass{scrartcl}
 %Warning: I use a modified vc script so this will not compile unless you remove the \GITTag from the footer or download the modified version from my website.
\immediate\write18{./vc}
\input{vc}
\usepackage[T1]{fontenc}
\usepackage{scrpage2}
\pagestyle{scrheadings}
\setkomafont{pagefoot}{\small\normalfont\sffamily}
\ifoot[\GITTag\ \VCDateISO]{\GITTag\ \VCDateISO}
\ofoot[\GITAbrHash]{\GITAbrHash}
\usepackage{microtype}
\usepackage{url}
\usepackage{libertine}
\usepackage{inconsolata}
\author{Seamus Bradley}
\title{The colourchange package}
\usepackage{listings}
\lstloadlanguages{[LaTeX]TeX} % I need to tell listings how I want code typeset. Here come some options:
\lstset{% 
    basicstyle=\ttfamily\small, 
    commentstyle=\itshape\ttfamily\small, 
    showspaces=false, 
    showstringspaces=false, 
    breaklines=true, 
    breakautoindent=true, 
    captionpos=t,
    frame=single,
    escapeinside={(*}{*)},
} 
\begin{document}
\maketitle

\vspace{-2\normalbaselineskip}
\begin{center}
  \small \url{tex@seamusbradley.net}\\
  \url{http://www.seamusbradley.net/tex.php}
\end{center}

\section{Introduction}

The \lstinline+colourchange+ package provides options for changing the colour of beamer's structural elements.
There are two ways to do this: manual and automatic.

\section{Manual}

The command \lstinline+\selectmanualcolour+ sets the colour to its argument.
It accepts any named colour understood by \lstinline+xcolor+.

\section{Automatic}

There are two options that automatically change the colour: \lstinline+slidechange+ and \lstinline+framechange+.
Pass one of these to the package as an option and the colour of the structure will slowly change from one colour to another.
To set what colours the transition should be between, use: \lstinline+\selectcolourchange{first}{second}+ which makes the colour transition from
  \lstinline+first+ to \lstinline+second+ over the course of the presentation.

You will need to compile twice to see the effect since \lstinline+colourchange+ needs to know how many slides there are so it knows how far
  through the presentation it is at a given point.

\section{Usage}
I have been testing this by calling it as a standard package, I'm not sure how \lstinline+\usecolortheme+ deals with package options.
So if you want to use the manual colour changing option, all you need is: \lstinline+\usepackage{colourchange}+.
If you want one of the automatic colourchange options call \lstinline+\usepackage[slidechange]{colourchange}+ or 
  \lstinline+\usepackage[framechange]{colourchange}+.

You can pass the option \lstinline+defaultstyle+ to the package, this sets up the structural elements (the inner and outer theme) so that they use the colours.
Otherwise, you can use Beamer's own \lstinline+\useinnnertheme+ and \lstinline+\useoutertheme+ to use the themes Beamer defines.
Call inner and outer themes separately.
Some ``all inclusive'' themes call a colour theme themselves and this can lead to only some elements changing colour.

Use the \lstinline+draft+ option to turn off the colour transitions.

\section{Known problems}

The \lstinline+smoothbars+ and \lstinline+smoothtrees+ outer themes do some funky stuff with shading and colours, so they aren't currently supported.
All other outer and inner themes looked alright.
Testing was, however, pretty basic: not all combinations were tested.

\section{How it works}

It's a pretty simple package really.
Just like (some) basic beamer colour themes it defines all the components of the inner and outer themes 
  in terms of one ``master colour'' called \lstinline+structure+.
Then all it does is use various ways of changing this colour.
The manual option just gives you a way of changing this colour on the fly.
The two automatic ways basically fiddle with beamer slide and frame creation internals to set the structure colour whenever a new frame/slide is created.
See the \lstinline+.sty+ file for details.

\end{document}

%%% Local Variables: 
%%% mode: latex
%%% TeX-master: t
%%% End: 
