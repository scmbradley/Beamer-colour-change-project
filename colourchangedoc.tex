\documentclass{scrartcl}
 %Warning: I use a modified vc script so this will not compile unless you remove the \GITTag from the footer or download the modified version from my website.
\immediate\write18{./vc}
\input{vc}
\usepackage{scrpage2}
\pagestyle{scrheadings}
\setkomafont{pagefoot}{\small\normalfont\sffamily}
\ifoot[\VCDateISO\GITTag]{\VCDateISO\GITTag}
\ofoot[\GITAbrHash]{\GITAbrHash}

\usepackage{url}
\author{Seamus Bradley}

\title{The colourchange package}
\usepackage{listings}
\lstloadlanguages{[LaTeX]TeX} % I need to tell listings how I want code typeset. Here come some options:
\lstset{% 
    basicstyle=\ttfamily\small, 
    commentstyle=\itshape\ttfamily\small, 
    showspaces=false, 
    showstringspaces=false, 
    breaklines=true, 
    breakautoindent=true, 
    captionpos=t,
    frame=single,
    escapeinside={(*}{*)},
} 
\begin{document}
\maketitle

\begin{center}
  \small \url{tex@seamusbradley.net}\\
  \url{http://www.seamusbradley.net/tex.php}
\end{center}

\section{Introduction}

The \lstinline+colourchange+ package provides options for changing the colour of beamer's structural elements.
There are two ways to do this: manual and automatic.

\section{Manual}

The command \lstinline+\selectmanualcolour+ sets the colour to its argument.
It accepts any named colour understood by \lstinline+xcolor+.

\section{Automatic}

There are two options that automatically change the colour: \lstinline+slidechange+ and \lstinline+framechange+.
Pass one of these to the package as an option and the colour of the structure will slowly change from one colour to another.
To set what colours the transition should be between, use: \lstinline+\selectcolourchange{first}{second}+ which makes the colour transition from
  \lstinline+first+ to \lstinline+second+ over the course of the presentation.

\section{Usage}

You can pass the option \lstinline+defaultstyle+ to the package, this sets up the structural elements (the inner and outer theme) so that they use the colours.
Otherwise, you can use Beamer's own \lstinline+\useinnnertheme+ and \lstinline+\useoutertheme+ to use the themes Beamer defines.\footnote{%
  Some themes don't work so well yet: smoothbars, split, shadow, tree, smoothtree}
Call inner and outerthemes separately.
Some ``all inclusive'' themes call a colour theme themselves and this can lead to only some elements changing colour.

Use the \lstinline+draft+ option to turn off the colour transitions.

\end{document}

%%% Local Variables: 
%%% mode: latex
%%% TeX-master: t
%%% End: 
